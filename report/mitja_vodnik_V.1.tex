\documentclass[a4paper]{article}
\usepackage[slovene]{babel}
\usepackage[utf8]{inputenc}
\usepackage[T1]{fontenc}
%\usepackage[margin=2cm, bottom=3cm, foot=1.5cm]{geometry}
\usepackage{float}
\usepackage{graphicx}
\usepackage{amsmath}
\usepackage{amssymb}
\usepackage{subcaption}
\usepackage{hyperref}

\newcommand{\tht}{\theta}
\newcommand{\Tht}{\Theta}
\newcommand{\dlt}{\delta}
\newcommand{\eps}{\epsilon}
\newcommand{\thalf}{\frac{3}{2}}
\newcommand{\ddx}[1]{\frac{d^2#1}{dx^2}}
\newcommand{\ddr}[2]{\frac{\partial^2#1}{\partial#2^2}}
\newcommand{\mddr}[3]{\frac{\partial^2#1}{\partial#2\partial#3}}

\newcommand{\der}[2]{\frac{d#1}{d#2}}
\newcommand{\pder}[2]{\frac{\partial#1}{\partial#2}}
\newcommand{\half}{\frac{1}{2}}
\newcommand{\forth}{\frac{1}{4}}
\newcommand{\q}{\underline{q}}
\newcommand{\p}{\underline{p}}
\newcommand{\x}{\underline{x}}
\newcommand{\liu}{\hat{\mathcal{L}}}
\newcommand{\bigO}[1]{\mathcal{O}\left( #1 \right)}
\newcommand{\pauli}{\mathbf{\sigma}}
\newcommand{\bra}[1]{\langle#1|}
\newcommand{\ket}[1]{|#1\rangle}
\newcommand{\id}[1]{\mathbf{1}_{2^{#1}}}
\newcommand{\tinv}{\frac{1}{\tau}}
\newcommand{\s}{\sigma}
\newcommand{\us}{\underline{\s}}
\newcommand{\vs}{\vec{\s}}
\newcommand{\vr}{\vec{r}}
\newcommand{\vq}{\vec{q}}
\newcommand{\vv}{\vec{v}}
\newcommand{\vo}{\vec{\omega}}
\newcommand{\uvs}{\underline{\vs}}
\newcommand{\expected}[1]{\left\langle #1 \right\rangle}
\newcommand{\D}{\Delta}

\begin{document}

    \title{\sc\large Višje računske metode\\
		\bigskip
		\bf\Large Matrično produktni nastavki}
	\author{Mitja Vodnik, 28182041}
            \date{\today}
	\maketitle

    Z metodo Monte-Carlo obravnavamo primer 1D kvantnega oscilatorja v potencialu $V(q) = \half q^2 + \lambda q^4$.
    Zanimata nas harmonski ($\lambda = 0$) in anharmonski ($\lambda = 1$) primer. \\

    \begin{equation}\label{eq1}
        P_{q_1, q_2} \propto \exp \left( -\frac{M}{2\beta}(q_2 - q_1)^2 - \frac{\beta}{M}V(q_1) \right)
    \end{equation}

    \section{Biparticija osnovnega stanja - SVD} 

    \begin{figure}
        \centering
        \includegraphics[width = \textwidth]{slika1.pdf}
        \caption{Simetrična biparticija.}
        \label{slika1}
    \end{figure}

    \begin{figure}
        \centering
        \includegraphics[width = \textwidth]{slika2.pdf}
        \caption{Nekompaktna biparticija s periodo 1.}
        \label{slika2}
    \end{figure}

    \begin{figure}
        \centering
        \includegraphics[width = \textwidth]{slika3.pdf}
        \caption{Odvisnost od periode biparticije.}
        \label{slika3}
    \end{figure}

    \section{Biparticija stanja - MPA}

    \begin{figure}
        \centering
        \begin{subfigure}{\textwidth}
            \includegraphics[width = \textwidth]{slika4a.pdf}
        \end{subfigure}
        \begin{subfigure}{\textwidth}
            \includegraphics[width = \textwidth]{slika4b.pdf}
        \end{subfigure}
        \caption{Odvisnost od velikosti blokov - PRP.}
        \label{slika4}
    \end{figure}

    \begin{figure}
        \centering
        \begin{subfigure}{\textwidth}
            \includegraphics[width = \textwidth]{slika5a.pdf}
        \end{subfigure}
        \begin{subfigure}{\textwidth}
            \includegraphics[width = \textwidth]{slika5b.pdf}
        \end{subfigure}
        \caption{Odvisnost od velikosti blokov - PRP.}
        \label{slika5}
    \end{figure}

    \begin{figure}
        \centering
        \begin{subfigure}{\textwidth}
            \includegraphics[width = \textwidth]{slika6a.pdf}
        \end{subfigure}
        \begin{subfigure}{\textwidth}
            \includegraphics[width = \textwidth]{slika6b.pdf}
        \end{subfigure}
        \caption{Odvisnost od velikosti blokov - PRP.}
        \label{slika6}
    \end{figure}

\end{document}
