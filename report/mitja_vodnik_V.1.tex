\documentclass[a4paper]{article}
\usepackage[slovene]{babel}
\usepackage[utf8]{inputenc}
\usepackage[T1]{fontenc}
%\usepackage[margin=2cm, bottom=3cm, foot=1.5cm]{geometry}
\usepackage{float}
\usepackage{graphicx}
\usepackage{amsmath}
\usepackage{amssymb}
\usepackage{subcaption}
\usepackage{hyperref}
\usepackage{dirtytalk}

\newcommand{\tht}{\theta}
\newcommand{\Tht}{\Theta}
\newcommand{\dlt}{\delta}
\newcommand{\eps}{\epsilon}
\newcommand{\thalf}{\frac{3}{2}}
\newcommand{\ddx}[1]{\frac{d^2#1}{dx^2}}
\newcommand{\ddr}[2]{\frac{\partial^2#1}{\partial#2^2}}
\newcommand{\mddr}[3]{\frac{\partial^2#1}{\partial#2\partial#3}}

\newcommand{\der}[2]{\frac{d#1}{d#2}}
\newcommand{\pder}[2]{\frac{\partial#1}{\partial#2}}
\newcommand{\half}{\frac{1}{2}}
\newcommand{\forth}{\frac{1}{4}}
\newcommand{\q}{\underline{q}}
\newcommand{\p}{\underline{p}}
\newcommand{\x}{\underline{x}}
\newcommand{\liu}{\hat{\mathcal{L}}}
\newcommand{\bigO}[1]{\mathcal{O}\left( #1 \right)}
\newcommand{\pauli}{\mathbf{\sigma}}
\newcommand{\bra}[1]{\langle#1|}
\newcommand{\ket}[1]{|#1\rangle}
\newcommand{\id}[1]{\mathbf{1}_{2^{#1}}}
\newcommand{\tinv}{\frac{1}{\tau}}
\newcommand{\s}{\sigma}
\newcommand{\us}{\underline{\s}}
\newcommand{\vs}{\vec{\s}}
\newcommand{\vr}{\vec{r}}
\newcommand{\vq}{\vec{q}}
\newcommand{\vv}{\vec{v}}
\newcommand{\vo}{\vec{\omega}}
\newcommand{\uvs}{\underline{\vs}}
\newcommand{\expected}[1]{\left\langle #1 \right\rangle}
\newcommand{\D}{\Delta}

\newcommand{\range}[2]{#1, \ldots, #2}
\newcommand{\seq}[2]{#1 \ldots #2}
\newcommand{\psiCoef}[2]{\psi_{\range{#1}{#2}}}
\newcommand{\psiCoeff}[3]{\psi_{#1, \range{#2}{#3}}}
\newcommand{\mpa}[2]{#1^{(#2)}_{s_#2}}

\begin{document}

    \title{\sc\large Višje računske metode\\
		\bigskip
		\bf\Large Matrično produktni nastavki}
	\author{Mitja Vodnik, 28182041}
            \date{\today}
	\maketitle

    Bistvo te naloge je, da ustvarimo metodo za generiranje matrično produktnih nastavkov (MPA) za stanja verig spinov $\half$.
    To pomeni, da hočemo za neko stanje $\ket{\psi}$ verige dolžine $n$ generirati set matrik
    $\big\{\mpa{A}{j} | s_j \in \{0, 1\}, j = \range{1}{n}\big\}$, da velja:\\

    \begin{equation}\label{eq1}
        \ket{\psi} = \sum_{\range{s_1}{s_n}} \psiCoef{s_1}{s_n} \ket{\seq{s_1}{s_n}}
    \end{equation}

    \begin{equation}\label{eq2}
        \psiCoef{s_1}{s_n} = \prod_{j=1}^{n} \mpa{A}{j}
    \end{equation}

    ($s_j \in \{0, 1\}$ naj predstavlja orientacijo $z$ komponente spina na mestu $j$ - delamo v lastni bazi operatorja $S_z$)

    \section{MPA algoritem} 

    Predstavimo stanje $\ket{\psi}$ kot stolpec:

    \begin{equation}\label{eq3}
        \Psi^{(1)\prime} = \begin{pmatrix} \psiCoef{0}{0} \\ \vdots \\ \psiCoef{1}{1} \end{pmatrix}, \quad
            \Psi^{(1)\prime}_{(\seq{s_1}{s_n})} = \psiCoef{s_1}{s_n}
    \end{equation}

    Indeks elementa v stolpcu smo zapisali binarno, kot ustrezno konfiguracijo verige $(\seq{s_1}{s_n}) = (\seq{s_1}{s_n})_2$, tako da prvi spin
    predstavlja MSB, zadnji pa LSB zapisa. Stolpec sedaj razdelimo na pol in ustvarimo matriko z dvema vrsticama - iz binarnega zapisa indeksa
    \say{odščipnemo} MSB:

    \begin{equation}\label{eq4}
        \Psi^{(1)} = \begin{pmatrix} \seq{\psiCoeff{0}{0}{0}}{\psiCoeff{0}{1}{1}} \\ \seq{\psiCoeff{1}{0}{0}}{\psiCoeff{1}{1}{1}} \end{pmatrix},
            \quad \Psi^{(1)}_{(s_1)(\seq{s_2}{s_n})} = \psiCoef{s_1}{s_n}
    \end{equation}

    Na tako ustvarjeni mariki izvedemo SVD razcep:

    \begin{equation}\label{eq5}
        \Psi^{(1)} = U^{(1)} D^{(1)} V^{(1)\dagger},
            \quad \Psi^{(1)}_{(s_1)(\seq{s_2}{s_n})} = \sum_{k_1=1}^{M_1} U^{(1)}_{(s_1)k_1} \lambda^{(1)}_{k_1} V^{(1)\dagger}_{k_1(\seq{s_2}{s_n})}
    \end{equation}

    Prvi dve matriki iskanega seta dobimo kot vrstici matrike $U^{(1)}$:

    \begin{equation}\label{eq6}
        \begin{split}
            A^{(1)}_0 &= \left( \seq{U^{(1)}_{0, 1}}{U^{(1)}_{0, M_1}} \right) \\
            A^{(1)}_1 &= \left( \seq{U^{(1)}_{1, 1}}{U^{(1)}_{1, M_1}} \right)
        \end{split}
    \end{equation}

    Nato pripravimo matriko za naslednji korak:

    \begin{equation}\label{eq7}
        \Psi^{(2)\prime} = D^{(1)} V^{(1)\dagger},
            \quad \Psi^{(2)\prime}_{k_1(\seq{s_2}{s_n})} = \lambda^{(1)}_{k_1} V^{(1)\dagger}_{k_1(\seq{s_2}{s_n})}
    \end{equation}

    Opisani postopek sedaj posplošimo za nadaljne korake ($j = \range{2}{n - 1}$):

    \begin{enumerate}
        \item Začnemo z matriko $\Psi^{(j)\prime}$ dimenzije $M_{j-1} \times 2^{n-j+1}$. Najprej jo preoblikujemo v matriko dimenzije
            $2M_{j-1} \times 2^{n-j}$, tako da binarnemu zapisu drugega indeksa \say{odščipnemo} MSB:

            \begin{equation}\label{eq8}
                \Psi^{(j)}_{(k_{j-1}s_j)(\seq{s_{j+1}}{s_n})} = \Psi^{(j)\prime}_{k_{j-1}(\seq{s_{j}}{s_n})}
            \end{equation}

            (Uvedli smo oznako $(ab)$, ki pomeni, da binarna zapisa števil $a$ in $b$ združimo v eno število.)

        \item Na dobljeni matriki izvedemo SVD razcep:

            \begin{equation}\label{eq9}
                \begin{split}
                    \Psi^{(j)} &= U^{(j)} D^{(j)} V^{(1)\dagger}, \\
                    \quad
                    \Psi^{(j)}_{(k_{j-1}s_j)(\seq{s_{j+1}}{s_n})}
                    &= \sum_{k_j=1}^{M_j} U^{(j)}_{(k_{j-1}s_j)k_j} \lambda^{(j)}_{k_j} V^{(j)\dagger}_{k_j(\seq{s_{j+1}}{s_n})}
                \end{split}
            \end{equation}

        \item Naslednji matriki produktnega nastavka sestavimo po formuli:

            \begin{equation}\label{eq10}
                \left( A^{(j)}_{s_j} \right)_{k_{j-1}k_j} = U^{(j)}_{(k_{j-1}s_j)k_j}
            \end{equation}

            Formula nam pove, da sta matriki sestavljeni iz alternirajočh vrstic matrike $U^{(j)}$:

            \begin{equation}\label{eq11}
                \begin{split}
                    A^{(j)}_0 &= \begin{pmatrix}
                        U^{(j)}_{(10)1}       & \cdots & U^{(j)}_{(10)M_j}     \\
                        \vdots                &        & \vdots                 \\
                        U^{(j)}_{(M_{j-1}0)1} & \cdots & U^{(j)}_{(M_{j-1}0)M_j} \\
                    \end{pmatrix} \\
                    A^{(j)}_1 &= \begin{pmatrix}
                        U^{(j)}_{(11)1}       & \cdots & U^{(j)}_{(11)M_j}     \\
                        \vdots                &        & \vdots                 \\
                        U^{(j)}_{(M_{j-1}1)1} & \cdots & U^{(j)}_{(M_{j-1}1)M_j} \\
                    \end{pmatrix}
                \end{split}
            \end{equation}

        \item Pripravimo še matriko za naslednji korak:

            \begin{equation}\label{eq12}
                \Psi^{(j+1)\prime} = D^{(j)}V^{(j)\dagger}
            \end{equation}

    Zadnji matriki nastavka dobimo kot stolpca matrike dobljene v zadnjem koraku:
 
    \begin{equation}\label{eq13}
        A^{(n)}_0 = \begin{pmatrix} \Psi^{(n)}_{1(0)} \\ \vdots \\ \Psi^{(n)}_{M_{n-1}(0)} \end{pmatrix}
        A^{(n)}_1 = \begin{pmatrix} \Psi^{(n)}_{1(1)} \\ \vdots \\ \Psi^{(n)}_{M_{n-1}(1)} \end{pmatrix}
    \end{equation}
           
    \end{enumerate}

    \section{Biparticija osnovnega stanja - SVD} 

    \begin{figure}
        \centering
        \includegraphics[width = \textwidth]{slika1.pdf}
        \caption{Odvisnost entropije prepletenosti simetrične biparticije osnovnega stanja od dolžine verige.}
        \label{slika1}
    \end{figure}

    \begin{figure}
        \centering
        \includegraphics[width = \textwidth]{slika2.pdf}
        \caption{Odvisnost entropije prepletenosti alternirajoče biparticije osnovnega stanja s periodo 1 od dolžine verige.}
        \label{slika2}
    \end{figure}

    \begin{figure}
        \centering
        \includegraphics[width = \textwidth]{slika3.pdf}
        \caption{Odvisnost entropije prepletenosti od periode blokov alternirajoče biparticije.}
        \label{slika3}
    \end{figure}

    \section{Biparticija stanja - MPA}

    \begin{figure}
        \centering
        \begin{subfigure}{\textwidth}
            \includegraphics[width = \textwidth]{slika4a.pdf}
        \end{subfigure}
        \begin{subfigure}{\textwidth}
            \includegraphics[width = \textwidth]{slika4b.pdf}
        \end{subfigure}
        \caption{Odvisnost entropije prepletenosti od velikosti blokov biparticije osnovnega stanja s periodičnimi robnimi pogoji.}
        \label{slika4}
    \end{figure}

    \begin{figure}
        \centering
        \begin{subfigure}{\textwidth}
            \includegraphics[width = \textwidth]{slika5a.pdf}
        \end{subfigure}
        \begin{subfigure}{\textwidth}
            \includegraphics[width = \textwidth]{slika5b.pdf}
        \end{subfigure}
        \caption{Odvisnost entropije prepletenosti od velikosti blokov biparticije osnovnega stanja z odprtimi robnimi pogoji.}
        \label{slika5}
    \end{figure}

    \begin{figure}
        \centering
        \begin{subfigure}{\textwidth}
            \includegraphics[width = \textwidth]{slika6a.pdf}
        \end{subfigure}
        \begin{subfigure}{\textwidth}
            \includegraphics[width = \textwidth]{slika6b.pdf}
        \end{subfigure}
        \caption{Odvisnost entropije prepletenosti od velikosti blokov biparticije naključnega stanja.}
        \label{slika6}
    \end{figure}

\end{document}
